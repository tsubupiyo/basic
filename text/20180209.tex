\documentclass[a4j,12pt]{jarticle}
\bibliographystyle{junsrt}
\usepackage{amsmath}
\usepackage{amssymb}
%\usepackage{listing}
\usepackage[dvipdfmx]{graphicx}
\usepackage{url}
\usepackage{color}
\usepackage{listings}
\usepackage{latexsym}

\def\theequation{\thesection.\arabic{equation}}
\makeatletter
\@addtoreset{equation}{section}
\makeatother

\makeatletter
 \renewcommand{\thefigure}{%
   \thesection.\arabic{figure}}
  \@addtoreset{figure}{section}
\makeatother

\setlength{\topmargin}{20mm}
\addtolength{\topmargin}{-1in}
\setlength{\oddsidemargin}{20mm}
\addtolength{\oddsidemargin}{-1in}
\setlength{\evensidemargin}{15mm}
\addtolength{\evensidemargin}{-1in}
\setlength{\textwidth}{170mm}
\setlength{\textheight}{254mm}
\setlength{\headsep}{0mm}
\setlength{\headheight}{0mm}
\setlength{\topskip}{0mm}
\def\vector#1{\mbox{\boldmath $#1$}}
\lstdefinestyle{myCustomMatlabStyle}{
  language=C++,
  tabsize=3,
  showspaces=false,
  showstringspaces=false,
  frame=tb,
  keywordstyle={\color{red}},
  keywordstyle={[2]\color{green}},
  keywordstyle={[3]\color{pink}},
  emph={void,const,int,vector},
  emphstyle={\color{blue}},
  captionpos=b
}
\renewcommand{\refname}{文献リスト}

\title{\Large{膜班 —現状報告—}\\
\large{}}
\author{糸賀 響}
\date{\today}

\begin{document}
\maketitle \thispagestyle{empty}
%http://wwwcp.tphys.uni-heidelberg.de/biophysics/biophysics_lecture2.pdf
\section{ガウス鎖を潰してみたときのエントロピーのラフな計算}
\subsection{ガウス鎖について}
配向的に相関のない結合により作られるガウス鎖は、
結合を意味するリンクベクトルが長さ一定でなく、
確率分布 $G(\vector{r})$ を持つ。
\begin{eqnarray}
   G(\vector{r}) = 
   \left(\cfrac{3}{2\pi b^2}\right)^{3/2}{\rm exp}\left(-\cfrac{3\vector{r}^2}{2b^2}\right)\\
   \langle\vector{r}^2\rangle=b^2
\end{eqnarray}
末端間ベクトル $\vector{R}_{\rm e}$ の確率分布は、
\begin{eqnarray}
   P(\vector{R}_{\rm e})
   &=& P(\{\vector{r}_n\})\\
   &=& \prod_{n=1}^N\left(\cfrac{3}{2\pi b^2}\right)^{3/2}{\rm exp}\left(-\cfrac{3\vector{r}^2}{2b^2}\right)\\
   &=& \left(\cfrac{3}{2\pi b^2}\right)^{3/2}{\rm exp}\left(-\sum_{n=2}^N\cfrac{3(\vector{R}_{n}-\vector{R}_{n-1})^2}{2b^2}\right)
\end{eqnarray}
であるため、エントロピーは
\begin{eqnarray}
   S
   &=& \ln P = \sum_{n=1}^{N}\ln P(\vector{r}_n)\\
   &=& {\rm const} - \cfrac{3}{2b^2}\sum_{n=1}^N\vector{r_n^2}
\end{eqnarray}
となる。
そこで自由エネルギーは、
\begin{eqnarray}
   \label{eq:free}
   F(\{\vector{r}_n\}) = E + \cfrac{3T}{2b^2}\sum_{n=1}^N\vector{r_n^2}
\end{eqnarray}
となる。ここで $E$ は配置に依存しない内部エネルギーである。
\subsection{潰した時の自由エネルギー}
2つのガウス鎖が接近する時、重なる部分の自由度が $1$ になると大胆に考えてみる。
つまり球冠部分に含まれる架橋点を除いて、また距離の関数として式 \ref{eq:free} を求める。

%$x$軸上で接近するとして、
\begin{eqnarray}
   \Delta F(a) = \cfrac{3}{2}TN\left(1-\int_{\substack{r_1\in [-\infty:a]\\r_2,r_3\in[-\infty:+\infty]}}G(\vector{r}){\rm d}\vector{r}\right)
\end{eqnarray}

\end{document}

